\documentclass[11pt,a4paper]{report}
\usepackage[utf8]{inputenc}
\usepackage[activeacute,spanish]{babel}
\usepackage{amsmath}
\usepackage{amsfonts}
\usepackage{amssymb}
\usepackage{graphicx}
\usepackage[left=4cm,right=3cm,top=4cm,bottom=3cm]{geometry}
\usepackage{hyperref}
\usepackage{multicol}
\author{Claudio Carrillo, Martin Morice, Raul Flores. \\ Profesor: Jaime Alvarez---ayudante: Alexis Inzunza}

\title{Iinforme laboratorio 6}
\begin{document}
\begin{figure}[hbtp]
\includegraphics[scale=1]{../Escritorio/ad.png}
\end{figure}

\maketitle
\part{ruteo estatico}
	Para esta actividad se pide hacer un ruteo estatico con 4 routers, con la siguiente topologia
	
\includegraphics[width=13cm]{../Escritorio/ruteo_estatico/1.jpg} 
	
	Primero se deben configurar las ip de las interfaces del router (FastEthernet y serial). Lo que podemos hacer por consola o gráficamente. Primero configuraremos el FastEthernet:
	
\includegraphics[width=13cm]{../Escritorio/ruteo_estatico/routerconsola.jpg} 	

	para ethrnet elegimos
	
\includegraphics[width=13cm]{../Escritorio/ruteo_estatico/routergrafico.jpg}  

192.168.0.0 /24 como IP. Y para serial elegimos la IP 10.0.0.0 /30

Luego configuramos los puertos serial:

\includegraphics[width=13cm]{../Escritorio/ruteo_estatico/serialgrafico.jpg} 

\includegraphics[width=13cm]{../Escritorio/ruteo_estatico/serialconsola.jpg} 
  
Al terminar de configurar IP’s procedemos a hacer el ruteo de nuestra red. Para esto determinamos rápidamente cual podría ser la mejor ruta de una sub red a otra, por medio de saltos, y asi elegimos una ruta. Luego configuramos en cada uno de los routers, gráficamente o por consola, si se usa la consola el comando es ip route “network” “mask” “next loop”.

Primero configuramos la ruta a las subredes ethernet (IP:192.168.0.0/24)

\includegraphics[width=13cm]{../Escritorio/ruteo_estatico/Serial192.jpg} 
  
Y para finalizar ruteamos la red con IP 10.0.0.0/24.

\includegraphics[width=13cm]{../Escritorio/ruteo_estatico/Serial100.jpg}

Al finalizar la configuración de IP’s y Rutas, probamos la red mandando un mensaje del PC0 al PC4. Primero enviamos el mensaje sin hacer el ruteo y luego con el ruteo implementado.

\includegraphics[width=13cm]{../Escritorio/ruteo_estatico/Screenshot_1.jpg}

\part{ruteo dinamico}

Siguiendo las instruciones del lab configurumos el sistema de ruteo dinamico

\includegraphics[width=13cm]{../Escritorio/ruteo_dinamico/imaenes/1.png}

\includegraphics[width=13cm]{../Escritorio/ruteo_dinamico/imaenes/2.png} 

\includegraphics[width=13cm]{../Escritorio/ruteo_dinamico/imaenes/3.png} 
   
\includegraphics[width=13cm]{../Escritorio/ruteo_dinamico/imaenes/4.png} 
\end{document}