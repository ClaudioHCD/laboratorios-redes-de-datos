\documentclass[11pt,a4paper]{report}
\usepackage[utf8]{inputenc}
\usepackage[activeacute,spanish]{babel}
\usepackage{amsmath}
\usepackage{amsfonts}
\usepackage{amssymb}
\usepackage{graphicx}
\usepackage[left=4cm,right=3cm,top=4cm,bottom=3cm]{geometry}
\usepackage{hyperref}
\usepackage{multicol}
\author{Claudio Carrillo, Martin Morice, Raul Flores. \\ Profesor: Jaime Alvarez---ayudante: Alexis Inzunza}

\title{Iinforme laboratorio 3}
\begin{document}
\begin{figure}[hbtp]
\includegraphics[scale=1]{../Escritorio/ad.png}
\end{figure}

\maketitle
\part{instalacion de scapy y wireshark}
Para instalar whireshark en ubuntu es necesario abrir el terminal e ingresar el sigiente comando:\\ sudo apt-get  install wireshark

\includegraphics[width=13cm]{../Escritorio/wire.png} 

luego para instalar scapy en una terminal se ingresa el siguiente comando \\ sudo atp-get install python-scapy

\includegraphics[width=13cm]{../Escritorio/12.png} 
\part{creando un frame}
 procedemos a crear el frame
 
 lo primero que hicimos fue asignar una variable a igualandola a la funcion ether(), para asi poder manipularla y poder darle nuestros propios valores, despues creamos otra variable de nombre b y le asignamos IP(), luego asignamos la variable c (que simula la capa de trasnporte) a el comando ICMP(),en segimiento asignamos a raw() la variable d, y agreagmos la direcion de destino y salida a la variable a, y por ultimo unimos el todas las variables al frame en si a traves de el siguiente comndo: \\ paquetito=a/b/c/d.
   
 
\includegraphics[width=13cm]{../Escritorio/ajd.png} 

\part{enviado y recepcion}

luego en ocupamos el comando sendp para enviar el frame.

\includegraphics[width=13cm]{../Escritorio/pant.png} 

aqui precentamos el envio del paquete

\includegraphics[width=13cm]{../Escritorio/wireshare2.png} 

y aqui la recepcion de un paquete 

\includegraphics[width=13cm]{../Escritorio/wireshare1.png}
\section{por medio del hub}
al enviarse un frame por medio del hub este le manda una copia a todos los computadores conectados a el, y el que tenga la mac de destino o puede abrir en los demas les llega pero no los puedden leer.

\section{por medio del switch}
la principar diferencia del switch con el hub es que el switch no replica el mensaje a todos los servidores conectados, solo lo envia al servidor con la mac de destino.
  
\part{respuestas cuestionario}
1- Se produce un envio broadcast que significa que la informacion de el paquete sera enviada a todos los canales de la red.

2-Depende de si la red tiene como nucleo central un switch o un hub, si es un switch el paquete sera enviado directamente al pc que tenga la mac de destino, por que el switch tiene identificados los pc y sus respectivas ip y mac, por lo que envia directamente.

Por otro lado si fuera un hub se utilizaria algo parecido a un broadcast, se enviaria a todos los pc hasta encontrar al que sea poseedor de la mac de destino.

3-Si uno envia algo a una mac inexistente en el sistema, el frame se envia igual sin poder llegar a el destino estimado.

Cualquier persona facilmente crear una mascara con la mac de destino del frame anterior y asi interceptar dicho frame.
  

\end{document}
